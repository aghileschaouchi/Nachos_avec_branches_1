\documentclass[12pt,a4paper]{article}

\usepackage{fullpage}
\usepackage[utf8]{inputenc}
\usepackage{amsfonts}
\usepackage{amssymb}
\usepackage[french]{babel}
\usepackage[cyr]{aeguill}

\setlength{\parindent}{2em}
\setlength{\parskip}{1em}

\newcommand{\quotes}[1]{``#1''}

\begin{document}

\begin{titlepage}
\centering
{\scshape\LARGE Université de Bordeaux \par}
{\scshape\Large Master 1 Informatique \par}
\vspace{3cm}

{\Huge\bfseries Devoir Système d'Exploitation \par}
\vspace{0.5cm}
{\Large\itshape Rapport 1: NACHOS Entrées/Sorties \par}

\vfill
réalisé par \par
Quoc-Nghia \textsc{NGUYEN} \par
Aghiles \textsc{CHAOUCHI} \par
Nadhir \textsc{HOUARI} \par
\vfill

{\large 10 octobre 2016\par}

\end{titlepage}

\section{Bilan}
Pour ce premier devoir, nous avons mis en place sous Nachos \quotes{un mécanisme d'entrée-sortie minimal} en implémentant:
\begin{enumerate}
\item Une couche d'entrées-sorties synchrones (SynchConsole) au-dessus de la couche Console avec les fonctions \textbf{SynchPutChar(int ch)}, \textbf{SynchGetChar()}, \textbf{SynchPutString(const char s[])}, \textbf{SynchGetString()} qui encapsulent tout le mécanisme des sémaphores.

\item Les appels systèmes \textbf{PutChar(char c)}, \textbf{GetChar()}, \textbf{PutString(const char s[])}, \textbf{GetString(char *s, int n)}... qui permettent de passer l'exécution en mode noyau et profiter des fonctions d'entrée-sortie synchrones du SynchConsole.
\end{enumerate}

En suivant les indications du sujet et les documentations des fonctions similaires dans le monde Linux, nous avons enfin implémenté toutes les fonctionnalités mentionnées dans le sujet (sauf la $printf()$, car, franchement, nous n'arrivons pas à compiler NACHOS après l'ajout du vsprintf.c tandis qu'il ne nous restait pas beaucoup de temps...).

Ces implémentations semblent fonctionner correctement car elles passent nos jeux de test et rendent bien les résultats attendus. En outre, nous surveillons la fuite de mémoire en exécutant explicitement \textbf{valgrind --leak-check=full} pour chaque test afin d'assurer que \textit{nos implémentations} ne posent pas ce genre de problème.

\section{Points délicats}
Il y a deux parties qui nous ont pris du temps pour finir, ce sont:
\begin{enumerate}
\item Chercher le comportement \textit{correct} des fonctions de manipulation de la chaîne de caractères, surtout répondre à la question: le caractère nul \verb$`\0'$ est-il compté dans $n$ caractères qu'on doit lire ou écrire ?
\item Deviner le cycle de vie d'un programme d'utilisateur exécuté sous NACHOS.
\end{enumerate}

Pour résoudre le premier problème, à l'aide de la commande $man$ sous Linux, nous avons écrit les jeux de test avec les fonctions similaires en C telles que $strncpy$, $fgets$, $fputs$... En variant la taille de la chaîne source et la taille du buffer utilisé, nous avons constaté le comportement de ces fonctions dans plusieurs cas différents.

Nous avons enfin décidé que: soit n le nombre de caractères maximum qu'une fonction peut traiter, en tout cas cette fonction doit traiter au plus $n-1$ caractères et réserver toujours une place pour le caractère de terminaison \verb$`\0'$. Ainsi, \verb$`\0'$ ne sera pas compté comme caractère de la chaîne source. Cela permet aussi de garantir la sécurité du système car les chaînes de caractères traitées par nos fonctions entrées/sorties se terminent toujours par \verb$`\0'$.

Quant au deuxième problème, cela prend du temps pour que nous trouvions la routine \verb$__start$ qui initialise le programme d'utilisateur en appelant successivement $main()$ et puis $Exit(0)$ à la sortie de $main()$. Pour que cette routine puisse fonctionner correctement, il nous fallait implémenter l'appel système $Exit()$ qui prend en compte la valeur retournée par $main()$ et appelle explicitement $Halt()$ pour arrêter la machine. A l'aide des commentaires dans le code source fourni et des messages de debug rendus en exécutant \texttt{nachos -d m}, nous arrivons enfin à remplacer l'instruction \texttt{move \$4 \$0} par \texttt{move \$4 \$2} pour que la valeur de retour de $main()$ sera l'argument $arg1$ de $Exit()$.

D'ailleurs, nous avons rencontré aussi un petit problème concernant la taille de la chaîne suffisant pour stocker un int de 4 bytes. Comme 13 caractères semblent suffire pour 11 chiffres, le symbole moins et le \verb$`\0'$, nous avons décider d'y allouer 16 caractères. Cependant cela reste toujours une question de l'efficacité.

\section{Limitations}
Quelques avantages de notre implémentation:
\begin{itemize}
\item Simple et bien commentée
\item Suit l'implémentation des fonctions similaires en C
\item Fonctionne!
\end{itemize}

Quelques inconvénients:
\begin{itemize}
\item L'efficacité en question, en raison de l'algorithme non optimisé.
\end{itemize}

\section{Tests}
Comme le but de ce devoir est de mettre en place un système d'entrée/sortie, la plupart des tests qu'on doit réaliser sont de mettre en entrée un caractère ou une chaîne de caractères et de vérifier la sortie (l'affichage).

Vous trouverez donc dans le répertoire \verb$./test$ les tests correspondent aux appels systèmes $PutChar$, $GetChar$, $PutString$, $GetString$, $PutInt$, $GetInt$ qui peuvent être lancés directement depuis le répertoire \verb$code$ avec la commande \texttt{./userprog/nachos -x /test/nom\_du\_test}

Le stress test pourra être réalisé explicitement sur le test $getstring.C$ qui utilise les deux fonctions de manipulation de string $PutString$ et $GetString$ avec un buffer. Pour ce fait, nous avons fait comme suit:

\begin{enumerate}
\item Générer un fichier texte de taille très grande (ie. avec la commande \texttt{base64 /udev/random | head -c 10000000 > /tmp/input.txt})
\item Augmenter la taille du buffer dans $getstring.c$
\item Lancer le test depuis code: \texttt{./userprog/nachos -x test/getstring < /tmp/input.txt}
\end{enumerate}

Les résultats reçus sont tous corrects par rapport à ceux qu'on attend.
\end{document}

